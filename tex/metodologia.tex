\chapter{Metodologia}\label{chp:metodologia}

\section{Coq}
O assistente de provas Coq \cite{Coq2021} é um sistema gerenciador de desenvolvimentos formais, composto por três linguagens: Gallina (especificação), Vernacular (comandos) e Ltac (metaprogramação). O vernacular define comandos que permitem interagir com o ambiente de provas, como adicionar definições, realizar consultas e alterar configurações do assistente. Ltac é utilizado para metaprogramação, sendo principalmente utilizada no desenvolvimento de táticas e automação. Por último, Gallina é uma linguagem funcional (de especificação) com tipos dependentes, na qual é implementado o Cálculo de Construções (Co)Indutivas \cite{Coquand1988,Coquand1990,PaulinMohring1993}. Para o desenvolvimento de provas, através do isomorfismo de Curry-Howard \cite{Soerensen2006}, Coq permite construir termos Gallina (programas), interativamente, por meio da aplicação de táticas. Ao final, o termo construído é enviado ao \textit{kernel} do Coq para verificação. A seguir, destaco alguns tópicos mais relevantes ao desenvolvimento deste trabalho.

\section{Classes de tipos}\label{sec:classes}
As classes de tipos em Coq, são semelhantes a implementação em Haskell \cite{Hall1996}, isto é, um mecanismo de polimorfismo \textit{ad-hoc} \cite{Wadler1989} em que funções podem ser aplicadas a diferentes argumentos, dependendo do seu tipo. No assistente de provas Coq, classes de tipos são açúcar sintático (\textit{syntax sugar}) para registros paramétricos dependentes e
%similar a classes em C++ ou Java. 
contam, também, com um sistema de busca de provas e inferência, similar a Prolog. Por consequência, classes de tipos são cidadãos de primeira classe, dessa forma restrições de classe, são apenas parâmetros implícitos \cite{Sozeau2008}. Por exemplo, em Haskell, a classe de tipos ordenados depende da classe de tipos com igualdade \cite{HaskellOrd}, representado: \haskellinline{Eq a => Ord a}. Afim de obter o mesmo efeito em Coq, supondo a existência das mesmas classes, basta passarmos como parâmetros a classe \haskellinline{Eq} a classe \haskellinline{Ord}: \coqinline{Ord (a: Type) (e: Eq a)}. Logo, classes de tipos em Coq são mais poderosas que sua contrapartida em Haskell. 
%Outro exemplo é a possibilidade de se definir múltiplas instâncias para o mesmo tipo: 
%em Coq, pode-se definir múltiplas instâncias de monoide para os naturais, um para operações de soma outro para multiplicação. Em Haskell, o mesmo efeito só é possível definindo novos tipos (\haskellinline{Sum} e \haskellinline{Prod}) para cada instância de monoide \cite{HaskellMonoid}.

Classes de tipos trazem também uma nova possibilidade, mais simples, para a construção de uma hierarquia de estruturas algébricas. Dentre diversas implementações de sucesso variado \cite{Geuvers2002,CruzFilipe2004,Garillot2009,Cohen2020}. Spitters e van der Weegen \cite{Spitters2011} propõe justamente isto, definir uma hierarquia algébrica através de classes, com um detalhe: separação entre, no que eles chamam de \textit{unbundling}, classe operacionais e classes predicados. Classes operacionais permitem referenciar operações, como operadores binários em grupos e monoides, concedendo um nome e uma notação canônica. Enquanto classes predicadas agrupam propriedades, que no caso de grupos representam seus axiomas (veja seção~\ref{sec:exemplos}).

\section{Igualdade e \textit{setoids}}\label{sec:igualdade}
A igualdade padrão do Coq é sintática, definida pelo tipo indutivo \coqinline{eq} e representado pelo símbolo ``='':
\begin{coqcode}
Inductive eq (A: Type) (x: A): A → Prop := eq_refl: x = x.	
\end{coqcode}
O código acima define uma família de igualdades parametrizada pelo tipo \coqinline{A}. A única forma de construir uma prova \coqinline{a = b} é através do construtor \coqinline{eq_refl}, desde que \coqinline{a} e \coqinline{b} sejam intencionalmente, ou sintaticamente iguais. Igualdade intencional ocorre quando dois termos são convertíveis, isto é, são redutíveis entre sí, via regras de conversão do Coq: $\alpha$, $\beta$, $\iota$, $\delta$, $\zeta$ e $\eta$-expansão \cite{CoqConversion}.  Esta definição é bem útil para reescrita, como pode ser visto no princípio de indução/recursão gerado para \coqinline{eq}:
\begin{coqcode}
eq_rect: ∀ (A: Type) (x: A) (P: A → Type), P x → ∀ y: A, x = y → P y.
\end{coqcode}
pois permite reescrever termos, em qualquer contexto \coqinline{P}, desde que sejam intencionalmente iguais. 

Especificações e formalizações podem apresentar objetos sintaticamente diferentes, não-convertíveis, mas que representam o mesmo objeto semanticamente, identificados por meio de uma relação de equivalência, por exemplo termos $\alpha$-equivalentes do \lcalc. Para estes casos, não podemos aplicar a família de táticas de reescrita do Coq, pois utilizam \coqinline{eq_rect} em sua implementação, o que acaba restringindo seu uso apenas à igualdade sintática. Uma solução ingênua seria tentar definir conjuntos quocientes, entretanto não é possível sem a adição de axiomas extras \cite{Chicli2003}, além de tornar o algoritmo de checagem de tipos indecidível \cite{Geuvers2002}. A alternativa é utilizar \textit{setoids}.

Os \textit{setoids}, também conhecidos como conjuntos de Bishop \cite{Barthe2003,Bishop2012}, são estruturas formadas por um tipo equipado com uma relação de equivalência, geralmente implementados por meio de registros:
\begin{coqcode}
Record Setoid (A: Type) := {
   car: A; 
   eqv: relation A; 
   prf: Equivalence equiv
}
\end{coqcode}
O grande problema de \textit{setoids} é que na prática, o usuário tem que lidar constantemente com os detalhes da implementação. Por exemplo, supondo uma representação da classe de $\alpha$-equivalência para termos-$\lambda$, denotado por \coqinline{Λα}:
\begin{coqcode}
Definition Λα: Setoid Λ := {| car := Λ; eqv: α; prf := _ |}.
\end{coqcode}
com os símbolos \coqinline{Λ} e \coqinline{α} representando, respectivamente, o conjunto dos termos-$\lambda$ e a relação de $\alpha$-equivalência. 
%A notação \coqinline{{||}} é utilizada para construção de registros.
Uma função que extrai o corpo da abstração-$\lambda$, representado pelo construtor \coqinline{Lam}, pode ser definido como:
\begin{coqcode}
Definition extract (t: Λα): Λα :=
   let t' := t.car in
   match t with
   | Lam x b => {| car := b; eqv: t.eqv; prf := _ |}
   | _ => t.
\end{coqcode}
Observe que o usuário precisa extrair o tipo envolto no \textit{setoid}, já que não é possível realizar um casamento em tipos não indutivos, e reconstruir o \textit{setoid} no retorno da função. Este padrão de código, que envolve extração e reconstrução, se agrava quando há ocorrência de mais de um \textit{setoid}, conhecido como ``\textit{setoid hell}'', tornando-se bastante oneroso e complexo, pois introduz mais problemas do que soluções \cite{Altenkirch2017}.

\section{Classes de tipos e \textit{setoids}}\label{sec:classes-setoids}
Afim de mitigar algumas dessas deficiências descritas na seção anterior, Coq define \textit{setoids} como uma classe de tipo:
\begin{coqcode}
Class Setoid (A: Type) := {
   equiv: relation A ;
   setoid_equiv :> Equivalence equiv 
}.
\end{coqcode}
Dessa forma, o usuário não precisa lidar diretamente com a relação e sua prova de equivalência, pois o Coq aplica busca de provas afim de encontrar uma instância \coqinline{Setoid} adequada. Para resolver o problema da reescrita, introduzida na seção~\ref{sec:igualdade}, Coq possui uma implementação de reescrita generalizada \cite{Sozeau2009}, também conhecida como reescrita \textit{setoids}, que permite substituir termos equivalentes, isto é \textit{setoids}, através de uma interface simplificada. Para tanto, o usuário precisa informar sobre quais funções é seguro reescrever, em outras palavras, se a função é respeitosa, ou \textit{respectful}:
\begin{definicao}[Função Respeitosa]\label{def:respeitosa}
	Sejam $X$ e $Y$ tipos, com suas respectivas relações de equivalências $\approx_X$ e $\approx_Y$. Uma função $f: X \rightarrow Y$ é respeitosa, se esta preserva as relações de equivalência para todas as entradas:
	\begin{equation*}
		\forall a~b \in X,~a \approx_X b \rightarrow f(a) \approx_Y f(b)
	\end{equation*}
\end{definicao}\noindent
A próxima seção apresenta um conjunto de exemplos mais aprofundado, englobando classes de tipos, \textit{setoids} e reescrita generalizada.

%\fabricio{MANDAR PARA FORMALIZAÇÃO.}
%O emprego de \textit{setoids} neste projeto foi essencial para manter a formalização construtiva.
%A capacidade de representar o conjunto de termos $\alpha$-equivalentes, \textit{setoids} são essências para manter a formalização construtiva.
%Além do problema da reescrita, descrito acima, e  O axioma da extensionalidade funcional é necessário para demonstrar que o conjunto de funções é um conjunto de permutação (\fabricio{REFERENCIAL TEÓRICO}). Apesar do sistema formal do Coq ser consistente com o axioma, perde-se a construtividade, o que pode acarretar em atritos futuros, relacionados a extração de código verificado ou o uso da formalização como biblioteca. \textit{Setoids} permitem recuperar a extensionalidade funcional, mantendo a construtividade, através de uma nova relação de equivalência computacional para funções (\fabricio{VEJA FORMALIZAÇÃO.}).

\section{Exemplo de classes de tipo e reescrita generalizada}\label{sec:exemplos}
Com o intuito de exemplificar tudo que foi apresentado, abaixo apresento uma implementação da classe de tipos grupo:\\
\begin{minipage}[t]{0.5\linewidth}
\begin{coqcode*}{linenos=true}
Class Neutral A := neutral: A.
Class Operator A := op: A → A → A.
Class Inverse A := inv: A → A.

Class Group (A : Type)
  `{Ntr: Neutral A, Opr: Operator A, Inv: Inverse A, Equiv A} : Prop := {
	grp_setoid :> Equivalence (≡@{A});
	grp_op_proper :> Proper ((≡@{A}) ⇒ (≡@{A}) ⇒ (≡@{A})) (+);
	grp_inv_proper :> Proper ((≡@{A}) ⇒ (≡@{A})) (-);
	(* ... *)
	grp_left_inv : ∀ (x : A), (-x) + x ≡@{A} ɛ@{A};
}.
\end{coqcode*}
\end{minipage}
\hspace{-1em}
\begin{minipage}[t]{0.5\linewidth}
\begin{coqcode}
Notation ɛ := neutral.
Infix "+" := op.
Notation "- x" := (inv x).
\end{coqcode}
\end{minipage}\vspace*{.6em}\\
Linhas 1--3 são classes operacionais. O objetivo é dar um nome canônico, e notação, as operações e elementos de grupos. Temos o elemento neutro (\coqinline{neutral}  \coqinline{ɛ}), a operação binária (\coqinline{op}  \coqinline{+}) e função inversa (\coqinline{inv}  \coqinline{-}). Entre as linhas 5--12 tem-se a implementação da classe predicada \coqinline{Group} (a maior parte é omitida para simplificação da discussão), seu objetivo é agrupar os axiomas de grupo. A classe recebe cinco parâmetros, um explícito tipo \coqinline{A}, seguindo de quatro implícitos generalizados (entre \coqinline{`{}}): três instâncias das classes operacionais e por último uma instância da classe \coqinline{Equiv}. Esta última permite definir grupo para um \textit{setoid} de \coqinline{A}, separando em classe operacional, referenciada pelo nome \coqinline{equiv} com notação \coqinline{≡}, e classe predicada (\coqinline{Equiv}) contendo a relação de equivalência. Os parâmetros entre \coqinline{`{}}, como dito, são implicitamente generalizados, isto é, caso dependam de outros parâmetros implícitos, o Coq os generaliza e incluem na lista de parâmetros implícitos, simplificando o trabalho do usuário, pois este não precisa memorizar todos os parâmetros necessários a uma classe. Concluindo a parte sintática, as notações terminadas em \coqinline{@{A}} fornecem explicitamente parâmetros, que diferentemente seriam inferidos implicitamente. As linhas 7--11 são as propriedades da classe, onde apenas a 11 é referente aos axiomas de grupo (as demais foram omitidas). As propriedades entre 7--9 são necessárias a reescrita \textit{setoid}: \coqinline{grp_setoid} é uma prova de equivalência para a relação \coqinline{≡@{A}} (classe \coqinline{Equivalence}) e \coqinline{grp_op_proper} e \coqinline{grp_inv_proper} são provas de que a operação binária e inversão são próprias, isto é, contextos nos quais é seguro realizar reescrita generalizada. \coqinline{Proper} espera dois argumentos: uma assinatura e uma função. A assinatura descreve as relações de equivalência para as entradas e saída da função, isto é, \coqinline{Proper ((≡@{A}) ⇒ (≡@{B}) ⇒ (≡@{C})) g} é equivalente a:
\begin{equation*}\label{eq:proper}
	\forall (x y : A) (z w : B),~x \equiv_A y \rightarrow z \equiv_B w \rightarrow g(x,z) \equiv_C g(y,w)
\end{equation*}
Assim, o assistente sabe como reescrever \coqinline{x + y} para \coqinline{z + y}, dado uma prova de \coqinline{x ≡ z}. 

O mecanismo de reescrita generalizada simplifica bastante as provas, eliminando quase completamente as inconveniências envolvendo \textit{setoid}. Abaixo apresento um breve resultado utilizando as técnicas descritas acima, comparando uma propriedade de teoria de grupos e seu equivalente no assistente:
\vspace*{.4em}\\
%\hspace*{-1em}
\begin{minipage}[t]{0.5\linewidth}
	\begin{lema}
		Seja $x \in G$, tal que $G$ é um grupo. A função inversa é involutiva:
		\begin{align*}
			& & x& =\\
			\text{\textup{id. esquerda}}& & \varepsilon + x& =\\
			\text{\textup{inv. esquerda}}& & (-(-x) + -x) + x& =\\
			\text{\textup{associatividade}}& & -(-x) + (-x + x)& =\\
			\text{\textup{inv. esquerda}}& & -(-x) + \varepsilon& =\\
			\text{\textup{id. direita}}& & -(-x) &
		\end{align*}
	\end{lema}
\end{minipage}
\hspace{.7 em}
\begin{minipage}[t]{0.5\linewidth}
	\begin{coqcode}
Lemma grp_inv_involutive {G: Type}
  (x: G) `{Group G}: -(-x) ≡@{G} x.
Proof with auto.

		
rewrite <-(grp_left_id x) at 2;
rewrite <-grp_left_inv;
rewrite <-grp_assoc;
rewrite grp_left_inv;
rewrite grp_right_id...
		
Qed.
	\end{coqcode}
\end{minipage}\vspace*{.6em}\\
À esquerda tem-se a prova informal, enquanto à direita sua prova formal. O lema \coqinline{grp_inv_involutive} tem como parâmetros: um tipo \coqinline{G} (implícito), um termo \coqinline{x} de \coqinline{G} e uma prova implícita de que \coqinline{G} possui uma instância de grupo. Por estar implicitamente generalizada (entre \coqinline{`{}}), o Coq inclui, (implicitamente) todos os parâmetros de \coqinline{Group}. Portanto, temos acesso a um operador binário, uma função inversa, um elemento neutro e uma equivalência para \coqinline{G}, além das propriedades de grupo e reescrita nos operadores \coqinline{+} e \coqinline{-}. Sem o mecanismo de reescrita generalizada fornecido pelo Coq, a demonstração do lema seria, em sua maioria, aplicação de propriedades de infraestrutura, em outras palavras, 
permite uma formalização sucinta, próximo a prova informal, enquanto os detalhes ficam a cargo do assistente de provas.
%Como pode ser notado, não há necessidade de lidar diretamente com a implementação de \textit{setoids} e lemas de compatibilidade manualmente. Tornando a formalização quase transparente aos \textit{setoids}.