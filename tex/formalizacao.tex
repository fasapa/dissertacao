\chapter{Formalização}\label{chp:formalizacao}

\section{Átomos}
O que nos interessa de nomes são seus identificadores, cadeia de caracteres, estrutura interna é irrelevante, por isso o chamamos de átomos, também são conhecidos como ``nomes puros''. Nomes são indivisíveis (opacos), infinitos contáveis com igualdade decidível. Não pode haver dúvidas quando dois nomes são iguais ou diferentes. Dado um conjunto \textbf{qualquer} de nomes \textbf{sempre} podemos obter um novo.
% PQ TEM QUE SER CONTÁVEL?

\begin{coqcode}
	Class GAction `(Group G) (X : Type) `{Act : Action G X, Equiv X} : Prop := {
		gact_setoid :> Equivalence(≡@{X});
		gact_proper :> Proper ((≡@{G}) ==> (≡@{X}) ==> (≡@{X})) ();
		gact_id : ∀ (x: X), ε@{G} • x ≡@{X} x;
		gact_compat: ∀ (p q: G) (x: X), p  (q  x) ≡@{X} (q + p)  x
	}.
\end{coqcode}

\section{Copello}
\section{Pritam}