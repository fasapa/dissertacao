\chapter{Referencial Teórico}\label{chp:ref-teorico}

Neste capítulo apresentamos os conceitos e definições fundamentais utilizados neste trabalho. Começamos pelo conceito de nome, seguido de permutações, pilares da teoria nominal \cite{Gabbay2002,Pitts2003,Pitts2013}, e finalizamos com conjuntos nominais, também chamados de tipos nominais por alguns autores \cite{Urban2008,Choudhury2015}, incluindo o conceito de suporte.

Nomes são úteis por seus identificadores, assim como ponteiros em C/C++, ou referencias em Java ou C\#. Sua estrutura interna é abstrata, irrelevante e indivisível. Por isso são denominadas átomos. Assumimos um conjunto contável\footnote{Não é estritamente necessário que o conjunto de nomes seja contável, veja \cite[Exercício 6.2, página 109]{Pitts2013}.} infinito de nomes, com igualdade decidível, denominado $\nameset$, representado pelas letras minúsculas no início do alfabeto: $a$, $b$, etc. A teoria nominal captura a noção de (in)dependência de nomes através de uma metateoria, simples e uniforme, baseada em permutações de nomes.

As permutações são funções bijetoras, de um conjunto para si mesmo, na qual formam um grupo de simetria, através da composição de funções ($\circ$) como operação binária. Uma propriedade de permutações, é que estas podem ser decompostas em 2-ciclos, também chamados de transposição ou \textit{swap}:
\begin{definicao}[Transposição ou \textit{Swap}]\label{def:swap}
	Seja $X$ um conjunto, tal que $a~b \in X$. Transposições são permutações 2-ciclos, representando por $\langle a~b \rangle : X \rightarrow X$ e definido como:
	\begin{equation}
		\langle a~b \rangle ~c \triangleq 
		\begin{cases}
			b & \text{\normalfont se } a = c \\
			a & \text{\normalfont se } b = c \\
			c & \text{\normalfont caso contrário}
		\end{cases}
	\end{equation} 
\end{definicao}\noindent
Fazendo $X = \nameset$, obtemos a operação de transposição de nomes, essencialmente uma função que troca nomes. O interesse em utilizar transposição como operação básica de alteração sintática, em gramáticas com estruturas ligantes, ao contrário da substituição de termos, está no fato de não dependerem da definição de variável livre ou ligada e serem imunes a captura de variáveis livres \cite{Pitts2003}. 

Denotamos por $\permset = (\nameset, \circ)$ o grupo de permutações finitas sobre $\nameset$, formado pela composição de transposições, com a função identidade como elemento neutro. Reservamos as letras minúsculas $p$, $q$, $r$ e $s$ para designar membros deste grupo. Permutações agem sobre conjuntos por meio da operação de ação (à esquerda) de grupo:
\begin{definicao}[Ação de grupo \cite{Pitts2013}]
	Seja $(G, \circ)$ um grupo com elemento neutro representado por $\neutral$.  Define-se ação à esquerda de $G$ sobre um conjunto $X$ qualquer como uma função $(\action) : G \times X \rightarrow X$,
	%(\fabricio{é melhor: $(\bullet) : G \rightarrow X \rightarrow X$ ?}),
	que satisfaz:
	\begin{align}
		\forall x \in X,&~~ \neutral \action x = x \label{eq:act-neutral}\\
		\forall g~h \in G, \forall x \in X,&~~ g \action (h \action x) = (g \circ h) \action x \label{eq:act-compat}
	\end{align}
\end{definicao}\noindent
\begin{definicao}[Ação de permutação e conjunto de permutação \cite{Pitts2013}]\label{def:acao-permutacao}
	Ação de permutação é uma ação definida pelo grupo $\permset$. Conjunto de permutação é aquele dotado de uma ação de permutação.
\end{definicao}\noindent

Ao enriquecer um conjunto com uma ação de permutação (definição \ref{def:acao-permutacao}) conseguimos estabelecer uma noção de dependência de nomes, aonde o conjunto de nomes ao qual uma estrutura depende é chamado de \textbf{suporte}, enquanto seu complemento é denominado de nomes frescos.
Mas primeiramente, vamos discutir o significado informal de (in)dependência, no contexto do \lcalc, pois seu significado depende da gramática e semântica em questão.

O conjunto dos termos ($\Lambda$) do \lcalc~é dado pela gramática \cite{Hindley2008}, utilizando nomes no lugar de variáveis:
\begin{equation*}
	M, N \in \Lambda ::= a \in \nameset \mid (M N) \mid (\lambda a . M)
\end{equation*}
aonde estamos utilizando nomes no lugar de variáveis, $(M N)$ representa a aplicação de dois termos e $(\lambda a . M)$ denota uma abstração. O conjunto de variáveis livres é definido como:
\begin{definicao}[Variáveis livres e ligadas \cite{Hindley2008}]
	O conjunto de variáveis livres é definido recursivamente na estrutura de um termo pela função $\text{FV}: \Lambda \rightarrow \nameset$, para $a \in \nameset$ e $M, N \in \Lambda$:
	\begin{equation*}
		\text{FV}(a) = \{a\}, ~ \text{FV}(MN) = \text{FV}(M) \cup \text{FV}(N), ~
		\text{FV}(\lambda a . M) = \text{FV}(M) - \{a\}
	\end{equation*}
	O conjunto de variáveis ligadas é definido recursivamente na estrutura de um termo pela função $\text{BV}: \Lambda \rightarrow \nameset$, da seguinte maneira:
	\begin{equation*}
		\text{BV}(a) = \{\emptyset\}, ~ \text{BV}(MN) = \text{BV}(M) \cup \text{BV}(N), ~
		\text{BV}(\lambda a . M) = \text{BV}(M) \cup \{a\}
	\end{equation*}
\end{definicao}\noindent
Dizemos que termos são $\alpha$-equivalentes, denotado $M \aeq N$, se estes diferem apenas nas variáveis ligadas.

Um termo, por exemplo $W = \lambda x. (xy)$, depende de todas as variáveis ($\text{BV}(W) \cup \text{FV}(W)$) contidas nele, pois a modificação de qualquer nome implica num novo termo, sintaticamente diferente, como $\lambda x. (xz)$ ou $\lambda z. (zy)$. Entretanto, se considerarmos termos módulo $\alpha$-equivalência, o conjunto de nomes dependentes são apenas as variáveis livres ($\text{FV}(W)$), pois continuando com nosso exemplo: $\lambda x. (xy) \not\aeq \lambda x. (xz)$, mas $\lambda x. (xy) \aeq \lambda z. (zy)$. Formalmente definimos suporte como:
\begin{definicao}[Suporte]
	Seja $X$ um conjunto de permutação, então $S \subset \nameset$ suporta $x \in X$ se:
	\begin{equation}\label{eq:suporte}
		\forall p \in \permset, ((\forall a \in S, p \action a = a) \rightarrow p \action x = x)
	\end{equation}
\end{definicao}\noindent
Ou seja, se uma permutação $p$ não modifica os nomes do suporte $S$ de $x$, então essa mesma permutação não terá efeito em $x$. Ao contrário, caso a permutação não modifique algum elemento de $S$ então ela não modificará $x$. A elegância da definição está no fato de fornecer uma forma simples de especificar dependência de nomes, apenas através da noção de permutação de nomes.

Gostaríamos de ressaltar dois pontos extras sobre a definição de suporte: (1) é importante que seja sempre finito. Caso $S = \nameset$, em suporte, então não é possível obter novos nomes frescos, pois esgotaram-se as opções.
(2) Podem existir mais de um suporte ($S_1$ e $S_2$) para um único elemento. Nestes casos é possível mostrar que $S_1 \cap S_2$ também é um suporte \cite[Proposição 2.3]{Pitts2013}. O que nos leva a definição de menor suporte:
\begin{definicao}[Menor suporte]
	Seja $X$ um conjunto de permutação, o menor suporte é dado pela intersecção de todos os suportes de $x \in X$:
	\begin{equation}\label{eq:menor-suporte}
		\text{supp}(x) \triangleq \bigcap \left\{ S \mid S \text{ suporta } x \right\}
	\end{equation}	
\end{definicao}\noindent
e frescor, até agora utilizado informalmente:
\begin{definicao}[Frescor]
	Dado dois conjuntos de permutação $X$ e $Y$, dizemos que $y \in Y$ é fresco em $x \in X$, denotado por $y \# x$, se:
	\begin{equation}\label{eq:frescor-supp}
		\text{supp}(y) \cap \text{supp}(y) = \emptyset
	\end{equation}
	Como na maioria dos casos $Y = \nameset$, temos que:
	\begin{equation}\label{eq:frescor}
		y \# x \leftrightarrow y \notin \text{supp}(x)
	\end{equation}
\end{definicao}\noindent
Assim, temos todas as ferramentas necessárias para definir conjuntos nominais:
\begin{definicao}[Conjunto nominal]
	Conjunto nominal é um conjunto de permutação na qual todos elementos possuem menor suporte.
\end{definicao}

%\fabricio{FALAR SOBRE O PROBLEMA DA CONSTRUTIVIDADE VS MENOR SUPORTE AQUI OU NA FORMALIZAÇÃO?}