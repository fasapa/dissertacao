\chapter{Perspectivas Futuras}\label{chp:perspectivas}
Apresentamos neste projeto uma formalização em progresso de conjuntos nominais no assistente de provas Coq. Obtivemos provas mais enxutas, com algumas sendo uma cópia passo a passo de sua contrapartida informal, ao comparar com a formalização de Choudhury \cite{AgdaNominal}. Tudo isso, graças nossa especificação da teoria nominal através classes de tipos associado ao mecanismo de reescrita generalizada do Coq.

Para obtenção de uma biblioteca automatizada três passos são fundamentais: formalização de funções suportadas e abstração de nomes e derivação do princípio de indução/recursão $\alpha$-estrutural. Nos próximos seis meses, julho à dezembro de 2021, propomos tratar as formalizações citadas. Com isso nosso trabalho estará no mesmo patamar à \cite{Choudhury2015}. A partir de janeiro de 2022, entrando nos últimos meses deste projeto, pretendemos reimplementar as técnicas de \cite{Copello2016} utilizando a formalização de conjuntos nominais desenvolvida neste trabalho.

Em trabalhos futuros, seria interessante ir além da formalização e desenvolver um sistema de derivação automático do princípio de indução/recursão $\alpha$-estrutural. Pois nos últimos anos tem sido desenvolvidas diversos aparatos de metaprogramação no Coq, como o projeto MetaCoq \cite{Sozeau2020}, que utiliza a própria linguagem de especificação Gallina como linguagem de metaprogramação, ou Elpi \cite{Dunchev2015,Tassi2018} um dialeto de $\lambda$Prolog \cite{Miller2009} incorporado no Coq.