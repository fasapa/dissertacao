\chapter{Perspectivas Futuras}\label{chp:perspectivas}
Apresentamos neste projeto uma formalização em progresso de conjuntos nominais no assistente de provas Coq. Obtivemos provas mais enxutas, com algumas sendo uma cópia passo a passo de sua contrapartida informal, ao comparar com a formalização de Choudhury \footnote{\url{https://www.cl.cam.ac.uk/~amp12/agda/choudhury/html/README.html}}. Tudo isso, graças nossa especificação da teoria nominal através classes de tipos associado ao mecanismo de reescrita generalizada do Coq. A próximas etapas são fundamentais para obtenção de uma biblioteca automatizada. Entretanto, pelo pouco tempo restante no projeto, e baseado em nossas observações do andamento do mesmo, realisticamente conseguiríamos finalizar a formalização de abstração de nomes e funções suportadas. Em trabalhos futuros, seria interessante ir além da formalização e desenvolver um sistema de derivação automático do princípio de indução/recursão $\alpha$-estrutural. Pois nos últimos anos tem sido desenvolvidas diversos aparatos de metaprogramação no Coq, como o projeto MetaCoq \cite{Sozeau2020}, que utiliza a própria linguagem de especificação Gallina como linguagem de metaprogramação, ou Elpi \cite{Dunchev2015,Tassi2018} um dialeto de $\lambda$Prolog \cite{Miller2009} incorporado no Coq.