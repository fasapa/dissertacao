\chapter{Introdução}\label{chp:intro}

A representação de estruturas ligantes ainda é um problema em aberto na área de formalização de sistemas formais. Dentre as principais técnicas podemos citar: índices de deBruijn \cite{Bruijn1972}, \textit{locally nameless} \cite{Chargueraud2011}, nominal \cite{Gabbay2002} e abordagens de sintaxe abstrata de alta ordem \cite{Pfenning1988,Harper1993}. Aydemir \textit{et. al} \cite{Aydemir2007} e Ambal \cite{Ambal2020} \textit{et. al}, compararam formalizações, respectivamente do cálculo $\lambda$ e do cálculo $\pi$, utilizando os métodos citas, entretanto não encontraram claramente uma técnica superior. deBruijn é a metateoria com implementação mais simples, de acordo com ambos os trabalhos, mas necessita da maior quantidade de lemas auxiliares, algo já notado por Hirschkoff em \cite{Hirschkoff1997}: do total de 800 lemas, 600 eram relacionados a operações de índices. \textit{Locally nameless} melhora a situação dos índices, entretanto introduz termos mal formados, o que implica na necessidade da adição um predicado (filtro) de bem formado a todas as propriedades. Sintaxe abstrata de alta ordem utiliza funções da metalinguagem (assistente de provas) para representar ligantes, o que dificulta seu raciocínio, já que o assistente não pode discursar sobre propriedades de suas próprias abstrações. Já as técnicas nominal
apresenta como as mais promissora, porquê simplifica a formalização já que o desenvolvimento é próximo do informal, ao custo de uma implementação complexa.

Dado a elaborada metateoria das técnicas e conjuntos nominais \cite{Pitts2006,Pitts2013}, formalizações que utilizam de técnicas nominais \cite{Aydemir2007, Urban2008, Copello2016, Copello2018, Ambal2020} tendem a implementar apenas os aspectos fundamentais da teoria: permutação de nomes e ação de permutação. Pois esta é mais simples e apresenta um comportamento mais previsível do que a substituição padrão de termos, já que não depende da noção de variável livre ou ligada e não há como ocorrer captura de variável livre \cite{Choudhury2015}.

Especificamente em relação a implementações das técnicas nominais, como bibliotecas prontas para uso, para lógicas clássicas tem-se o pacote nominal\footnote{\url{https://nms.kcl.ac.uk/christian.urban/Nominal/}} \cite{Urban2008} do assistente de provas Isabelle/HOL\footnote{\url{https://isabelle.in.tum.de/}} \cite{Nipkow2002} se destaca, visto que é um sistema completo e automatizado, capaz de demonstrar formalmente provas como se tivesse utilizando a convenção da variável de Barendregt \cite[BVC]{Barendregt2012}. Porém, em lógicas construtivas as opções são nulas. Aydemir \textit{et. al.} propôs uma prova de conceito em \cite{Aydemir2007} em Coq, na qual uma ferramenta externa leria a descrição de uma sintaxe, para em seguida construir uma especificação nominal axiomatizada. Todavia supõem-se o abandono do método, em favor do \textit{locally nameless}, ao analisar as publicações posteriores pelo grupo de pesquisa. Em Agda temos os trabalhos de Copello \textit{et. al.} \cite{Copello2016,Copello2018} , no qual foram capazes de derivar um princípio de indução e um combinador de recursão $\alpha$-estrutural \cite{Pitts2006}, dependendo apenas de uma propriedade extra, que chamaram de $\alpha$-compatibilidade. Infelizmente o método consta apenas com uma implementação manual para uma restrição do cálculo $\lambda$, com pouco espaço para automação, no estado atual. E por último, Choudhury em sua dissertação de mestrado sob orientação de Pitts \cite{Choudhury2015}, apresenta uma formalização construtiva de conjuntos nominais em Agda, que apesar de não conter aplicação, mesmo assim é o desenvolvimento formal mais completo de conjuntos nominais.

Pelo estado atual da teoria nominal em assistentes de provas construtivos, propomos atacar a falta de uma biblioteca nominal, similar ao pacote nominal do Isabelle, no assistente de provas Coq. Neste trabalhos apresentamos uma formalização em progresso de conjuntos nominais. Se as técnicas nominais permitem a formalização mais próxima das provas informais, gostaríamos de argumentar que, formalizar conjuntos nominais em Coq, permite também alcançar desenvolvimento sucintos da metateoria, similares à metateoria informal de conjuntos nominais \cite{Pitts2013}. Graças aos mecanismos de classes de tipos e reescrita generalizadas do Coq.

Este projeto está organizado da seguinte forma: capítulo~\ref{chp:ref-teorico} apresenta o referencial teórico de conjuntos nominais. Uma breve descrição do assistente de provas Coq, e dos mecanismos de classe (seção~\ref{sec:classes}) e reescrita (seção~\ref{sec:classes-setoids}) encontram-se no capítulo~\ref{chp:metodologia}. O capítulo~\ref{chp:formalizacao} apresenta nossas experimentações no desenvolvimento de uma biblioteca nominal: seção~\ref{sec:copello}
apresenta uma implementação da metodologia de Copello \textit{et. al.}, enquanto a seção~\ref{sec:choudhury} contem a formalização em progresso de conjuntos nominais, similar a \cite{Choudhury2015}. E por último o capítulo~\ref{chp:perspectivas}, apresentamos nossas conclusões, perspectivas futuras e um cronograma com etapas futuras do projeto.