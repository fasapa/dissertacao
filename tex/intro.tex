\chapter{Introdução}\label{chp:intro}

A representação de estruturas ligantes ainda é um problema em aberto na área de implementação de linguagens de programação e verificação de sistemas formais. Dentre as principais técnicas podemos citar: índices de de Bruijn \cite{Bruijn1972}, \textit{locally nameless} \cite{Chargueraud2011}, nominal \cite{Gabbay2002} e abordagens de sintaxe abstrata de alta ordem \cite{Pfenning1988,Harper1993}. Aydemir \textit{et al.} \cite{Aydemir2008} e Ambal \cite{Ambal2020} \textit{et al.}, compararam formalizações, respectivamente do cálculo $\lambda$ e do cálculo $\pi$, utilizando os métodos citados, entretanto nenhuma das técnicas se sobressai as demais. os índices de de Bruijn tem a metateoria com implementação mais simples, de acordo com ambos os trabalhos, mas necessita da maior quantidade de lemas auxiliares, algo já notado por Hirschkoff em \cite{Hirschkoff1997}: do total de 800 lemas, 600 eram relacionados a operações de índices. \textit{Locally nameless} melhora a situação dos índices, porem com a necessidade da adição um predicado (filtro) de termo bem formado a todas as propriedades. Sintaxe abstrata de alta ordem utiliza funções da metalinguagem (assistente de provas) para representar ligantes, o que dificulta seu raciocínio, já que o assistente não pode discursar sobre propriedades de suas próprias abstrações. Já as técnicas nominal, de acordo com \cite{Ambal2020}, é mais simples de escrever lemas e definições, já que são fieis às informais, ao custo de uma infraestrutura complexa, disponível apenas em Isabelle/HOL \cite{Aydemir2008}. 

% apresenta como as mais promissora, porquê simplifica a formalização já que o desenvolvimento é próximo do informal, ao custo de uma implementação complexa.

Dado a elaborada metateoria das técnicas e conjuntos nominais \cite{Pitts2006,Pitts2013}, formalizações que utilizam de técnicas nominais \cite{Aydemir2007, Urban2008, Copello2016, Copello2018, Ambal2020} tendem a implementar apenas os aspectos fundamentais da teoria: permutação de nomes e ação de permutação. Pois ação de permutação e permutação de nomes em conjunto possuem definições mais simples e apresenta um comportamento mais previsível do que a substituição padrão de termos, já que não depende da noção de variável livre ou ligada e não há como ocorrer captura de variável livre \cite{Choudhury2015}.

Especificamente em relação a implementações das técnicas nominais, para lógicas clássicas, o pacote nominal \cite{IsabelleNominal, Urban2008} do assistente de provas Isabelle/HOL \cite{Nipkow2002} é considero estado da arte \cite{Pitts2016}, visto que é um sistema completo e automatizado. Porém, para lógicas construtivas as opções são bem inferiores. Aydemir \textit{et. al.} propôs uma prova de conceito em \cite{Aydemir2007} em Coq, na qual uma ferramenta externa leria a descrição de uma sintaxe, para em seguida construir uma especificação nominal axiomatizada. Todavia supõem-se o abandono do método, em favor do \textit{locally nameless}, ao analisar as publicações posteriores pelo grupo de pesquisa. Em Agda tem-se os trabalhos de Copello \textit{et. al.} \cite{Copello2016,Copello2018} , no qual foram capazes de derivar um princípio de indução e um combinador de recursão $\alpha$-estrutural \cite{Pitts2006}, dependendo apenas de condição extra, que chamaram de $\alpha$-compatibilidade (Definição \ref{def:alpha-compatibilidade}). Infelizmente o método consta apenas com uma implementação manual para uma restrição do cálculo $\lambda$, com pouco espaço para automação, no estado atual. Por último, Choudhury em sua dissertação de mestrado sob orientação de Pitts \cite{Choudhury2015}, apresenta uma formalização construtiva de conjuntos nominais em Agda que, apesar de não ser aplicada, é o desenvolvimento formal mais completo de conjuntos nominais.

Pelo estado da arte atual da teoria nominal e sua aplicação em assistentes de prova
em assistentes de provas construtivos, como observado por Andrew Pitts em entrevista no recebimento do prêmio Alonzo Church 2019 \cite{Aceto2019}, propomos abordar a falta de uma biblioteca nominal, similar ao pacote nominal do Isabelle, no assistente de provas Coq. Neste trabalho apresentamos uma formalização em progresso de conjuntos nominais. Se as técnicas nominais permitem a formalização mais próxima das provas informais, gostaríamos de argumentar que, formalizar conjuntos nominais em Coq, permite também alcançar desenvolvimento sucintos da metateoria, similares à metateoria informal de conjuntos nominais \cite{Pitts2013}, graças aos mecanismos de classes de tipos e reescrita generalizadas do Coq.

Este projeto está organizado da seguinte forma: o Capítulo~\ref{chp:ref-teorico} apresenta o referencial teórico de conjuntos nominais. Uma breve descrição do assistente de provas Coq, e dos mecanismos de classe (Seção~\ref{sec:classes}) e reescrita (Seção~\ref{sec:classes-setoids}) encontram-se no Capítulo~\ref{chp:metodologia}. O Capítulo~\ref{chp:formalizacao} apresenta nossas experimentações no desenvolvimento de uma biblioteca nominal: a Seção~\ref{sec:copello}
apresenta uma implementação da metodologia de Copello \textit{et. al.}, enquanto a Seção~\ref{sec:choudhury} contém a formalização em progresso de conjuntos nominais, similar a \cite{Choudhury2015}. E por último no Capítulo~\ref{chp:perspectivas}  apresentamos um cronograma com etapas futuras do projeto e nossas conclusões e perspectivas futuras. 