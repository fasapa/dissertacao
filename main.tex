% !TeX encoding = UTF-8
% !TeX spellcheck = pt_BR
% !TeX root = main.tex

\documentclass[
	% -- opções da classe memoir --
    draft,
	12pt,				% tamanho da fonte
	%openright,			% capítulos começam em pág ímpar (insere página vazia caso preciso)
	%twoside,			% para impressão em recto e verso. Oposto a oneside
	oneside,
	a4paper,			% tamanho do papel.
	% -- opções da classe abntex2 --
	%chapter=TITLE,		% títulos de capítulos convertidos em letras maiúsculas
	%section=TITLE,		% títulos de seções convertidos em letras maiúsculas
	%subsection=TITLE,	% títulos de subseções convertidos em letras maiúsculas
	%subsubsection=TITLE,% títulos de subsubseções convertidos em letras maiúsculas
	% -- opções do pacote babel --
	english,			% idioma adicional para hifenização
	%french,				% idioma adicional para hifenização
	%spanish,			% idioma adicional para hifenização
	brazil				% o último idioma é o principal do documento
	]{abntex2}

% ---
% Pacotes básicos
% ---
\usepackage{fontspec}
\setmainfont{Latin Modern Roman}
\setmonofont{DejaVu Sans Mono}
%\usepackage{lmodern}			% Usa a fonte Latin Modern
%\usepackage[T1]{fontenc}		% Selecao de codigos de fonte.
%\usepackage[utf8]{inputenc}		% Codificacao do documento (conversão automática dos acentos)
\usepackage{indentfirst}		% Indenta o primeiro parágrafo de cada seção.
\usepackage{color}				% Controle das cores
\usepackage{graphicx}			% Inclusão de gráficos
\usepackage{microtype} 			% para melhorias de justificação
\usepackage{lipsum}
% ---

% ---
% Pacotes de citações
% ---
\usepackage[brazilian,hyperpageref]{backref}	 % Paginas com as citações na bibl
\usepackage[num]{abntex2cite}	% Citações padrão ABNT

% ---
% CONFIGURAÇÕES DE PACOTES
% ---

% ---
% Configurações do pacote backref
% Usado sem a opção hyperpageref de backref
\renewcommand{\backrefpagesname}{Citado na(s) página(s):~}
% Texto padrão antes do número das páginas
\renewcommand{\backref}{}
% Define os textos da citação
\renewcommand*{\backrefalt}[4]{
	\ifcase #1 %
		Nenhuma citação no texto.%
	\or
		Citado na página #2.%
	\else
		Citado #1 vezes nas páginas #2.%
	\fi}%
% ---

% ---
% Informações de dados para CAPA e FOLHA DE ROSTO
% ---
\titulo{Uma Formalização da Teoria Nominal em Coq}
\autor{Fabrício Sanches Paranhos}
\local{Goiânia, Goiás}
\data{2022}
\orientador{Daniel Lima Ventura}
%\coorientador{Equipe \abnTeX}
\instituicao{%
  Universidade Federal de Goiás -- UFG
  \par
  Instituto de Informática
  \par
  Programa de Pós-Graduação em Ciência da Computação}
\tipotrabalho{Dissertação (Mestrado)}
% O preambulo deve conter o tipo do trabalho, o objetivo,
% o nome da instituição e a área de concentração
\preambulo{Modelo canônico de trabalho monográfico acadêmico em conformidade com
as normas ABNT apresentado à comunidade de usuários \LaTeX.}
% ---


% ---
% Configurações de aparência do PDF final

% alterando o aspecto da cor azul
\definecolor{blue}{RGB}{41,5,195}

% informações do PDF
\makeatletter
\hypersetup{
     	%pagebackref=true,
		pdftitle={\@title},
		pdfauthor={\@author},
    	pdfsubject={\imprimirpreambulo},
	    pdfcreator={LuaLaTeX with abnTeX2},
		pdfkeywords={conjuntos nominais}{técnicas nominais}{formalização}{Coq},
		colorlinks=true,       		% false: boxed links; true: colored links
    	linkcolor=blue,          	% color of internal links
    	citecolor=blue,        		% color of links to bibliography
    	filecolor=magenta,      		% color of file links
		urlcolor=blue,
		bookmarksdepth=4
}
\makeatother
% ---

% ---
% Posiciona figuras e tabelas no topo da página quando adicionadas sozinhas
% em um página em branco. Ver https://github.com/abntex/abntex2/issues/170
\makeatletter
\setlength{\@fptop}{5pt} % Set distance from top of page to first float
\makeatother
% ---

% ---
% Possibilita criação de Quadros e Lista de quadros.
% Ver https://github.com/abntex/abntex2/issues/176
%
\newcommand{\quadroname}{Quadro}
\newcommand{\listofquadrosname}{Lista de quadros}

\newfloat[chapter]{quadro}{loq}{\quadroname}
\newlistof{listofquadros}{loq}{\listofquadrosname}
\newlistentry{quadro}{loq}{0}

% configurações para atender às regras da ABNT
\setfloatadjustment{quadro}{\centering}
\counterwithout{quadro}{chapter}
\renewcommand{\cftquadroname}{\quadroname\space}
\renewcommand*{\cftquadroaftersnum}{\hfill--\hfill}

\setfloatlocations{quadro}{hbtp} % Ver https://github.com/abntex/abntex2/issues/176
% ---

% ---
% Espaçamentos entre linhas e parágrafos
% ---

% O tamanho do parágrafo é dado por:
\setlength{\parindent}{1.3cm}

% Controle do espaçamento entre um parágrafo e outro:
\setlength{\parskip}{0.2cm}  % tente também \onelineskip

% ---
% compila o indice
% ---
\makeindex
% ---

% ---
% Pacotes extra USUÁRIO
% ---
% Matemática extra
\usepackage{amsmath,amssymb,amsthm}
\newtheorem{definicao}{Definição}
\newtheorem{corolario}{Corolário}

% Formatação de código
\usepackage{minted}
\usepackage{etoolbox}
\makeatletter
\AtBeginEnvironment{minted}{\dontdofcolorbox}
\def\dontdofcolorbox{\renewcommand\fcolorbox[4][]{##4}}
\makeatother
\newlength{\fancyvrbtopsep}
\newlength{\fancyvrbpartopsep}
\makeatletter
\FV@AddToHook{\FV@ListParameterHook}{\topsep=\fancyvrbtopsep\partopsep=\fancyvrbpartopsep}
\makeatother
\setlength{\fancyvrbtopsep}{0pt}
\setlength{\fancyvrbpartopsep}{0pt}

\newminted{coq}{autogobble=true, linenos=false, fontsize=\footnotesize, style=bw}
\newmintinline{coq}{fontsize=\footnotesize, style=bw}

% Macros e comandos extras
% Comentários
\newcommand{\cfabricio}[2]{\fabricio{#2}}
\newcommand{\fabricio}[1]{\textcolor{red} {#1}}
\newcommand{\ofabricio}[1]{\text{\sout{\fabricio{#1}}}}

% ----
% Início do documento
% ----
\begin{document}

% Seleciona o idioma do documento (conforme pacotes do babel)
%\selectlanguage{english}
\selectlanguage{brazil}

% Retira espaço extra obsoleto entre as frases.
\frenchspacing

% ----------------------------------------------------------
% ELEMENTOS PRÉ-TEXTUAIS
% ----------------------------------------------------------
\pretextual

% ---
% Capa
% ---
\imprimircapa
% ---

% ---
% Folha de rosto
% (o * indica que haverá a ficha bibliográfica)
% ---
\imprimirfolhaderosto*
% ---

% ---
% Inserir a ficha bibliografica
% ---

% Isto é um exemplo de Ficha Catalográfica, ou ``Dados internacionais de
% catalogação-na-publicação''. Você pode utilizar este modelo como referência.
% Porém, provavelmente a biblioteca da sua universidade lhe fornecerá um PDF
% com a ficha catalográfica definitiva após a defesa do trabalho. Quando estiver
% com o documento, salve-o como PDF no diretório do seu projeto e substitua todo
% o conteúdo de implementação deste arquivo pelo comando abaixo:
%
% \begin{fichacatalografica}
%     \includepdf{fig_ficha_catalografica.pdf}
% \end{fichacatalografica}

%\begin{fichacatalografica}
%	\sffamily
%	\vspace*{\fill}					% Posição vertical
%	\begin{center}					% Minipage Centralizado
%	\fbox{\begin{minipage}[c][8cm]{13.5cm}		% Largura
%	\small
%	\imprimirautor
%	%Sobrenome, Nome do autor
%
%	\hspace{0.5cm} \imprimirtitulo  / \imprimirautor. --
%	\imprimirlocal, \imprimirdata-
%
%	\hspace{0.5cm} \thelastpage p. : il. (algumas color.) ; 30 cm.\\
%
%	\hspace{0.5cm} \imprimirorientadorRotulo~\imprimirorientador\\
%
%	\hspace{0.5cm}
%	\parbox[t]{\textwidth}{\imprimirtipotrabalho~--~\imprimirinstituicao,
%	\imprimirdata.}\\
%
%	\hspace{0.5cm}
%		1. Palavra-chave1.
%		2. Palavra-chave2.
%		2. Palavra-chave3.
%		I. Orientador.
%		II. Universidade xxx.
%		III. Faculdade de xxx.
%		IV. Título
%	\end{minipage}}
%	\end{center}
%\end{fichacatalografica}
% ---

% ---
% Inserir errata
% ---
%\begin{errata}
%Elemento opcional da \citeonline[4.2.1.2]{NBR14724:2011}. Exemplo:
%
%\vspace{\onelineskip}
%
%FERRIGNO, C. R. A. \textbf{Tratamento de neoplasias ósseas apendiculares com
%reimplantação de enxerto ósseo autólogo autoclavado associado ao plasma
%rico em plaquetas}: estudo crítico na cirurgia de preservação de membro em
%cães. 2011. 128 f. Tese (Livre-Docência) - Faculdade de Medicina Veterinária e
%Zootecnia, Universidade de São Paulo, São Paulo, 2011.
%
%\begin{table}[htb]
%\center
%\footnotesize
%\begin{tabular}{|p{1.4cm}|p{1cm}|p{3cm}|p{3cm}|}
%  \hline
%   \textbf{Folha} & \textbf{Linha}  & \textbf{Onde se lê}  & \textbf{Leia-se}  \\
%    \hline
%    1 & 10 & auto-conclavo & autoconclavo\\
%   \hline
%\end{tabular}
%\end{table}
%
%\end{errata}
% ---

% ---
% Inserir folha de aprovação
% ---

% Isto é um exemplo de Folha de aprovação, elemento obrigatório da NBR
% 14724/2011 (seção 4.2.1.3). Você pode utilizar este modelo até a aprovação
% do trabalho. Após isso, substitua todo o conteúdo deste arquivo por uma
% imagem da página assinada pela banca com o comando abaixo:
%
% \begin{folhadeaprovacao}
% \includepdf{folhadeaprovacao_final.pdf}
% \end{folhadeaprovacao}
%
%\begin{folhadeaprovacao}
%
%  \begin{center}
%    {\ABNTEXchapterfont\large\imprimirautor}
%
%    \vspace*{\fill}\vspace*{\fill}
%    \begin{center}
%      \ABNTEXchapterfont\bfseries\Large\imprimirtitulo
%    \end{center}
%    \vspace*{\fill}
%
%    \hspace{.45\textwidth}
%    \begin{minipage}{.5\textwidth}
%        \imprimirpreambulo
%    \end{minipage}%
%    \vspace*{\fill}
%   \end{center}
%
%   Trabalho aprovado. \imprimirlocal, 24 de novembro de 2012:
%
%   \assinatura{\textbf{\imprimirorientador} \\ Orientador}
%   \assinatura{\textbf{Professor} \\ Convidado 1}
%   \assinatura{\textbf{Professor} \\ Convidado 2}
%   %\assinatura{\textbf{Professor} \\ Convidado 3}
%   %\assinatura{\textbf{Professor} \\ Convidado 4}
%
%   \begin{center}
%    \vspace*{0.5cm}
%    {\large\imprimirlocal}
%    \par
%    {\large\imprimirdata}
%    \vspace*{1cm}
%  \end{center}
%
%\end{folhadeaprovacao}
% ---

% ---
% Dedicatória
% ---
\begin{dedicatoria}
   \vspace*{\fill}
   \centering
   \noindent
   \textit{ Este trabalho é dedicado às crianças adultas que,\\
   quando pequenas, sonharam em se tornar cientistas.} \vspace*{\fill}
\end{dedicatoria}
% ---

% ---
% Agradecimentos
% ---
\begin{agradecimentos}
Os agradecimentos principais são direcionados à Gerald Weber, Miguel Frasson,
Leslie H. Watter, Bruno Parente Lima, Flávio de Vasconcellos Corrêa, Otavio Real
Salvador, Renato Machnievscz\footnote{Os nomes dos integrantes do primeiro
projeto abn\TeX\ foram extraídos de
\url{http://codigolivre.org.br/projects/abntex/}} e todos aqueles que
contribuíram para que a produção de trabalhos acadêmicos conforme
as normas ABNT com \LaTeX\ fosse possível.

Agradecimentos especiais são direcionados ao Centro de Pesquisa em Arquitetura
da Informação\footnote{\url{http://www.cpai.unb.br/}} da Universidade de
Brasília (CPAI), ao grupo de usuários
\emph{latex-br}\footnote{\url{http://groups.google.com/group/latex-br}} e aos
novos voluntários do grupo
\emph{\abnTeX}\footnote{\url{http://groups.google.com/group/abntex2} e
\url{http://www.abntex.net.br/}}~que contribuíram e que ainda
contribuirão para a evolução do \abnTeX.

\end{agradecimentos}
% ---

% ---
% Epígrafe
% ---
\begin{epigrafe}
   \vspace*{\fill}
	\begin{flushright}
		\textit{``Não vos amoldeis às estruturas deste mundo, \\
		mas transformai-vos pela renovação da mente, \\
		a fim de distinguir qual é a vontade de Deus: \\
		o que é bom, o que Lhe é agradável, o que é perfeito.\\
		(Bíblia Sagrada, Romanos 12, 2)}
	\end{flushright}
\end{epigrafe}
% ---

% ---
% RESUMOS
% ---

% resumo em português
\setlength{\absparsep}{18pt} % ajusta o espaçamento dos parágrafos do resumo
\begin{resumo}
	Apresentamos uma formalização em progresso de conjuntos nominais em Coq, salientando as principais decisões de design e projeto. Implementamos e comparamos dois métodos, um pragmático, apoiada nos pilares básicos da teoria nominal, e uma formalização construtiva de conjuntos nominais, propriamente dita. Mostramos que graças a classe de tipos e reescrita generalizada alcançamos definições e provas concisas, ao mesmo tempo evitando o conhecido cenário ``\textit{setoid hell}''.

 \textbf{Palavras-chave}: conjuntos nominais, técnicas nominais, formalização, Coq.
\end{resumo}

% resumo em inglês
%\begin{resumo}[Abstract]
% \begin{otherlanguage*}{english}
%   This is the english abstract.
%
%   \vspace{\onelineskip}
%
%   \noindent
%   \textbf{Keywords}: latex. abntex. text editoration.
% \end{otherlanguage*}
%\end{resumo}
% ---

% ---
% inserir lista de ilustrações
% ---
%\pdfbookmark[0]{\listfigurename}{lof}
%\listoffigures*
%\cleardoublepage
% ---

% ---
% inserir lista de quadros
% ---
%\pdfbookmark[0]{\listofquadrosname}{loq}
%\listofquadros*
%\cleardoublepage
% ---

% ---
% inserir lista de tabelas
% ---
%\pdfbookmark[0]{\listtablename}{lot}
%\listoftables*
%\cleardoublepage
% ---

% ---
% inserir lista de abreviaturas e siglas
% ---
%\begin{siglas}
%  \item[ABNT] Associação Brasileira de Normas Técnicas
%  \item[abnTeX] ABsurdas Normas para TeX
%\end{siglas}
% ---

% ---
% inserir lista de símbolos
% ---
%\begin{simbolos}
%  \item[$ \Gamma $] Letra grega Gama
%  \item[$ \Lambda $] Lambda
%  \item[$ \zeta $] Letra grega minúscula zeta
%  \item[$ \in $] Pertence
%\end{simbolos}
% ---

% ---
% inserir o sumario
% ---
\pdfbookmark[0]{\contentsname}{toc}
\tableofcontents*
\cleardoublepage
% ---



% ----------------------------------------------------------
% ELEMENTOS TEXTUAIS
% ----------------------------------------------------------
\textual

\chapter{Introdução}\label{chp:intro}
Ola Pits \cite{Pitts2013}
\chapter{Referencial Teórico}\label{chp:ref-teorico}

Neste capítulo apresentamos os conceitos e definições fundamentais utilizados neste trabalho. Começamos pelo conceito de nome, seguido de permutações, pilares da teoria nominal \cite{Gabbay2002,Pitts2003,Pitts2013}, e finalizamos com conjuntos nominais, também chamados de tipos nominais por alguns autores \cite{Urban2008,Choudhury2015}, incluindo o conceito de suporte.

Nomes são úteis por seus identificadores, assim como ponteiros em C/C++, ou referencias em Java ou C\#. Sua estrutura interna é abstrata, irrelevante e indivisível. Por isso são denominadas átomos. Assumimos um conjunto contável\footnote{Não é estritamente necessário que o conjunto de nomes seja contável, veja \cite[Exercício 6.2, página 109]{Pitts2013}.} infinito de nomes, com igualdade decidível, denominado $\nameset$, representado pelas letras minúsculas no início do alfabeto: $a$, $b$, etc. A teoria nominal captura a noção de (in)dependência de nomes através de uma metateoria, simples e uniforme, baseada em permutações de nomes.

As permutações são funções bijetoras, de um conjunto para si mesmo, na qual formam um grupo de simetria, através da composição de funções ($\circ$) como operação binária. Uma propriedade de permutações, é que estas podem ser decompostas em 2-ciclos, também chamados de transposição ou \textit{swap}:
\begin{definicao}[Transposição ou \textit{Swap}]\label{def:swap}
	Seja $X$ um conjunto, tal que $a~b \in X$. Transposições são permutações 2-ciclos, representando por $\langle a~b \rangle : X \rightarrow X$ e definido como:
	\begin{equation}
		\langle a~b \rangle ~c \triangleq 
		\begin{cases}
			b & \text{\normalfont se } a = c \\
			a & \text{\normalfont se } b = c \\
			c & \text{\normalfont caso contrário}
		\end{cases}
	\end{equation} 
\end{definicao}\noindent
Fazendo $X = \nameset$, obtemos a operação de transposição de nomes, essencialmente uma função que troca nomes. O interesse em utilizar transposição como operação básica de alteração sintática, em gramáticas com estruturas ligantes, ao contrário da substituição de termos, está no fato de não dependerem da definição de variável livre ou ligada e serem imunes a captura de variáveis livres \cite{Pitts2003}. 

Denotamos por $\permset = (\nameset, \circ)$ o grupo de permutações finitas sobre $\nameset$, formado pela composição de transposições, com a função identidade como elemento neutro. Reservamos as letras minúsculas $p$, $q$, $r$ e $s$ para designar membros deste grupo. Permutações agem sobre conjuntos por meio da operação de ação (à esquerda) de grupo:
\begin{definicao}[Ação de grupo \cite{Pitts2013}]
	Seja $(G, \circ)$ um grupo com elemento neutro representado por $\neutral$.  Define-se ação à esquerda de $G$ sobre um conjunto $X$ qualquer como uma função $(\action) : G \times X \rightarrow X$,
	%(\fabricio{é melhor: $(\bullet) : G \rightarrow X \rightarrow X$ ?}),
	que satisfaz:
	\begin{align}
		\forall x \in X,&~~ \neutral \action x = x \label{eq:act-neutral}\\
		\forall g~h \in G, \forall x \in X,&~~ g \action (h \action x) = (g \circ h) \action x \label{eq:act-compat}
	\end{align}
\end{definicao}\noindent
\begin{definicao}[Ação de permutação e conjunto de permutação \cite{Pitts2013}]\label{def:acao-permutacao}
	Ação de permutação é uma ação definida pelo grupo $\permset$. Conjunto de permutação é aquele dotado de uma ação de permutação.
\end{definicao}\noindent

Ao enriquecer um conjunto com uma ação de permutação (Definição \ref{def:acao-permutacao}) conseguimos estabelecer uma noção de dependência de nomes, aonde o conjunto de nomes ao qual uma estrutura depende é chamado de \textbf{suporte}, enquanto seu complemento é denominado de nomes frescos.
Mas primeiramente, vamos discutir o significado informal de (in)dependência, no contexto do \lcalc, pois seu significado depende da gramática e semântica em questão.

O conjunto dos termos ($\Lambda$) do \lcalc~é dado pela gramática \cite{Hindley2008}, utilizando nomes no lugar de variáveis:
\begin{equation}\label{eq:lambda}
	M, N \in \Lambda ::= a \in \nameset \mid (M N) \mid (\lambda a . M)
\end{equation}
aonde estamos utilizando nomes no lugar de variáveis, $(M N)$ representa a aplicação de dois termos e $(\lambda a . M)$ denota uma abstração. O conjunto de variáveis livres é definido como:
\begin{definicao}[Variáveis livres e ligadas \cite{Hindley2008}]
	O conjunto de variáveis livres é definido recursivamente na estrutura de um termo pela função $\text{FV}: \Lambda \rightarrow \nameset$, para $a \in \nameset$ e $M, N \in \Lambda$:
	\begin{equation*}
		\text{FV}(a) = \{a\}, ~ \text{FV}(MN) = \text{FV}(M) \cup \text{FV}(N), ~
		\text{FV}(\lambda a . M) = \text{FV}(M) - \{a\}
	\end{equation*}
	O conjunto de variáveis ligadas é definido recursivamente na estrutura de um termo pela função $\text{BV}: \Lambda \rightarrow \nameset$, da seguinte maneira:
	\begin{equation*}
		\text{BV}(a) = \{\emptyset\}, ~ \text{BV}(MN) = \text{BV}(M) \cup \text{BV}(N), ~
		\text{BV}(\lambda a . M) = \text{BV}(M) \cup \{a\}
	\end{equation*}
\end{definicao}\noindent
Dizemos que termos são $\alpha$-equivalentes, denotado $M \aeq N$, se estes diferem apenas nas variáveis ligadas.

Um termo, por exemplo $W = \lambda x. (xy)$, depende de todas as variáveis ($\text{BV}(W) \cup \text{FV}(W)$) contidas nele, pois a modificação de qualquer nome implica num novo termo, sintaticamente diferente, como $\lambda x. (xz)$ ou $\lambda z. (zy)$. Entretanto, se considerarmos termos módulo $\alpha$-equivalência, o conjunto de nomes dependentes são apenas as variáveis livres ($\text{FV}(W)$), pois continuando com nosso exemplo: $\lambda x. (xy) \not\aeq \lambda x. (xz)$, mas $\lambda x. (xy) \aeq \lambda z. (zy)$. Formalmente definimos suporte como:
\begin{definicao}[Suporte]
	Seja $X$ um conjunto de permutação, então $S \subset \nameset$ suporta $x \in X$ se:
	\begin{equation}\label{eq:suporte}
		\forall p \in \permset, ((\forall a \in S, p \action a = a) \rightarrow p \action x = x)
	\end{equation}
\end{definicao}\noindent
Ou seja, se uma permutação $p$ não modifica os nomes do suporte $S$ de $x$, então essa mesma permutação não terá efeito em $x$. Ao contrário, caso a permutação não modifique algum elemento de $S$ então ela não modificará $x$. A elegância da definição está no fato de fornecer uma forma simples de especificar dependência de nomes, apenas através da noção de permutação de nomes.

Gostaríamos de ressaltar dois pontos extras sobre a definição de suporte: (1) é importante que seja sempre finito. Caso $S = \nameset$, em suporte, então não é possível obter novos nomes frescos, pois esgotaram-se as opções.
(2) Podem existir mais de um suporte ($S_1$ e $S_2$) para um único elemento. Nestes casos é possível mostrar que $S_1 \cap S_2$ também é um suporte \cite[Proposição 2.3]{Pitts2013}. O que nos leva a definição de menor suporte:
\begin{definicao}[Menor suporte]
	Seja $X$ um conjunto de permutação, o menor suporte é dado pela intersecção de todos os suportes de $x \in X$:
	\begin{equation}\label{eq:menor-suporte}
		\text{supp}(x) \triangleq \bigcap \left\{ S \mid S \text{ suporta } x \right\}
	\end{equation}	
\end{definicao}\noindent
e frescor, até agora utilizado informalmente:
\begin{definicao}[Frescor]
	Dado dois conjuntos de permutação $X$ e $Y$, dizemos que $y \in Y$ é fresco em $x \in X$, denotado por $y \# x$, se:
	\begin{equation}\label{eq:frescor-supp}
		\text{supp}(y) \cap \text{supp}(y) = \emptyset
	\end{equation}
	Como na maioria dos casos $Y = \nameset$, temos que:
	\begin{equation}\label{eq:frescor}
		y \# x \leftrightarrow y \notin \text{supp}(x)
	\end{equation}
\end{definicao}\noindent
Assim, temos todas as ferramentas necessárias para definir conjuntos nominais:
\begin{definicao}[Conjunto nominal]\label{def:conjunto-nominal}
	Conjunto nominal é um conjunto de permutação na qual todos elementos possuem menor suporte.
\end{definicao}

%\fabricio{FALAR SOBRE O PROBLEMA DA CONSTRUTIVIDADE VS MENOR SUPORTE AQUI OU NA FORMALIZAÇÃO?}
\chapter{Metodologia}\label{chp:metodologia}

\section{Coq}
O assistente de provas Coq \cite{Coq2021} é um sistema gerenciador de desenvolvimentos formais, composto por três linguagens: Gallina (especificação), Vernacular (comandos) e Ltac (metaprogramação). O vernacular define comandos que permitem interagir com o ambiente de provas, como adicionar definições, realizar consultas e alterar configurações do assistente. Ltac é utilizado para metaprogramação, sendo principalmente utilizada no desenvolvimento de táticas e automação. Por último, Gallina é uma linguagem funcional (de especificação) com tipos dependentes, na qual é implementado o Cálculo de Construções (Co)Indutivas \cite{Coquand1988,Coquand1990,PaulinMohring1993}. Para o desenvolvimento de provas, através do isomorfismo de Curry-Howard \cite{Soerensen2006}, Coq permite construir termos Gallina (programas), interativamente, por meio da aplicação de táticas. Ao final, o termo construído é enviado ao \textit{kernel} do Coq para verificação. A seguir, destaco alguns tópicos mais relevantes ao desenvolvimento deste trabalho.

\section{Classes de tipos}\label{sec:classes}
As classes de tipos em Coq, são semelhantes a implementação em Haskell \cite{Hall1996}, isto é, ambos um mecanismo de polimorfismo \textit{ad-hoc} \cite{Wadler1989} em que funções podem ser aplicadas a diferentes argumentos, dependendo do seu tipo. No assistente de provas Coq, classes de tipos são açúcar sintático (\textit{syntax sugar}) para registros paramétricos dependentes e
%similar a classes em C++ ou Java. 
contam, também, com um sistema de busca de provas e inferência, similar a Prolog. Por consequência, as classes de tipos são cidadãs de primeira classe, dessa forma restrições de classe são apenas parâmetros implícitos \cite{Sozeau2008}. Por exemplo, em Haskell, a classe de tipos ordenados depende da classe de tipos com igualdade \cite{HaskellOrd}, representado: \haskellinline{Eq a => Ord a}. Afim de obter o mesmo efeito em Coq, supondo a existência das mesmas classes, basta passarmos como parâmetros a classe \haskellinline{Eq} a classe \haskellinline{Ord}: \coqinline{Ord (a: Type) (e: Eq a)}. Logo, classes de tipos em Coq são mais poderosas que sua contrapartida em Haskell. 
%Outro exemplo é a possibilidade de se definir múltiplas instâncias para o mesmo tipo: 
%em Coq, pode-se definir múltiplas instâncias de monoide para os naturais, um para operações de soma outro para multiplicação. Em Haskell, o mesmo efeito só é possível definindo novos tipos (\haskellinline{Sum} e \haskellinline{Prod}) para cada instância de monoide \cite{HaskellMonoid}.

Classes de tipos trazem também uma nova possibilidade, mais simples, para a construção de uma hierarquia de estruturas algébricas. Dentre diversas implementações de sucesso variado \cite{Geuvers2002,CruzFilipe2004,Garillot2009,Cohen2020}. Spitters e van der Weegen \cite{Spitters2011} propõe justamente isto, definir uma hierarquia algébrica através de classes, com um detalhe: separação entre, no que eles chamam de \textit{unbundling}, classe operacionais e classes predicados. Classes operacionais permitem referenciar operações, como operadores binários em grupos e monoides, concedendo um nome e uma notação canônica. Enquanto classes predicadas agrupam propriedades, que no caso de grupos representam seus axiomas (veja seção~\ref{sec:exemplos}).

\section{Igualdade e \textit{setoids}}\label{sec:igualdade}
A igualdade padrão do Coq é sintática, definida pelo tipo indutivo \coqinline{eq} e representado pelo símbolo ``='':
\begin{coqcode}
Inductive eq (A: Type) (x: A): A → Prop := eq_refl: x = x.	
\end{coqcode}
O código acima define uma família de igualdades parametrizada pelo tipo \coqinline{A}. A única forma de construir uma prova \coqinline{a = b} é através do construtor \coqinline{eq_refl}, desde que \coqinline{a} e \coqinline{b} sejam intencionalmente, ou sintaticamente iguais \cite{Chlipala2013}. Igualdade intencional ocorre quando dois termos são convertíveis, isto é, são redutíveis entre sí, via regras de conversão do Coq: $\alpha$, $\beta$, $\iota$, $\delta$, $\zeta$ e $\eta$-expansão \cite{CoqConversion}.  Esta definição é bem útil para reescrita, como pode ser visto no princípio de indução/recursão gerado para \coqinline{eq}:
\begin{coqcode}
eq_rect: ∀ (A: Type) (x: A) (P: A → Type), P x → ∀ y: A, x = y → P y.
\end{coqcode}
pois permite reescrever termos, em qualquer contexto \coqinline{P}, desde que sejam intencionalmente iguais. 

Especificações e formalizações podem apresentar objetos sintaticamente diferentes, não-convertíveis, mas que representam o mesmo objeto semanticamente, identificados por meio de uma relação de equivalência, por exemplo termos $\alpha$-equivalentes do \lcalc. Para estes casos, não podemos aplicar a família de táticas de reescrita do Coq, pois utilizam \coqinline{eq_rect} em sua implementação, o que acaba restringindo seu uso apenas à igualdade sintática. Uma solução ingênua seria tentar definir conjuntos quocientes, entretanto não seria possível sem a adição de axiomas extras \cite{Chicli2003}, além de tornar o algoritmo de checagem de tipos indecidível \cite{Geuvers2002}. A alternativa é utilizar \textit{setoids}.

Os \textit{setoids}, também conhecidos como conjuntos de Bishop \cite{Barthe2003,Bishop2012}, são estruturas formadas por um tipo equipado com uma relação de equivalência, geralmente implementados por meio de registros:
\begin{coqcode}
Record Setoid (A: Type) := {
   car: A; 
   eqv: relation A; 
   prf: Equivalence equiv
}
\end{coqcode}
O grande problema de \textit{setoids} é que, na prática, o usuário tem que lidar constantemente com os detalhes da implementação. Por exemplo, supondo uma representação da classe de $\alpha$-equivalência para termos-$\lambda$, denotada por \coqinline{Λα}:
\begin{coqcode}
Definition Λα: Setoid Λ := {| car := Λ; eqv: α; prf := _ |}.
\end{coqcode}
com os símbolos \coqinline{Λ} e \coqinline{α} representando, respectivamente, o conjunto dos termos-$\lambda$ e a relação de $\alpha$-equivalência. 
%A notação \coqinline{{||}} é utilizada para construção de registros.
Uma função que extrai o corpo da abstração-$\lambda$, representado pelo construtor \coqinline{Lam}, pode ser definido como:
\begin{coqcode}
Definition extract (t: Λα): Λα :=
   let t' := t.car in
   match t with
   | Lam x b => {| car := b; eqv: t.eqv; prf := _ |}
   | _ => t.
\end{coqcode}
Observe que o usuário precisa extrair o tipo envolto no \textit{setoid}, já que não é possível realizar um casamento em tipos não-indutivos, e reconstruir o \textit{setoid} no retorno da função. Este padrão de código, que envolve extração e reconstrução, se agrava quando há ocorrência de mais de um \textit{setoid}, conhecido como ``\textit{setoid hell}'', tornando-se bastante oneroso e complexo, pois introduz mais problemas do que soluções \cite{Altenkirch2017}.

\section{Classes de tipos e \textit{setoids}}\label{sec:classes-setoids}
Afim de mitigar algumas dessas deficiências descritas na seção anterior, Coq define \textit{setoids} como uma classe de tipo:
\begin{coqcode}
Class Setoid (A: Type) := {
   equiv: relation A ;
   setoid_equiv :> Equivalence equiv 
}.
\end{coqcode}
Dessa forma, o usuário não precisa lidar diretamente com a relação e sua prova de equivalência, pois o Coq aplica busca de provas afim de encontrar uma instância \coqinline{Setoid} adequada. Para resolver o problema da reescrita, introduzida na seção~\ref{sec:igualdade}, Coq possui uma implementação de reescrita generalizada \cite{Sozeau2009}, também conhecida como reescrita \textit{setoids}, que permite substituir termos equivalentes, isto é \textit{setoids}, através de uma interface simplificada. Para tanto, o usuário precisa informar sobre quais funções é seguro reescrever, em outras palavras, se a função é respeitosa, ou \textit{respectful}:
\begin{definicao}[Função Respeitosa]\label{def:respeitosa}
	Sejam $X$ e $Y$ tipos, com suas respectivas relações de equivalências $\approx_X$ e $\approx_Y$. Uma função $f: X \rightarrow Y$ é respeitosa, se esta preserva as relações de equivalência para todas as entradas:
	\begin{equation*}
		\forall a~b \in X,~a \approx_X b \rightarrow f(a) \approx_Y f(b)
	\end{equation*}
\end{definicao}\noindent
A próxima seção apresenta um conjunto de exemplos mais aprofundados, englobando classes de tipos, \textit{setoids} e reescrita generalizada.

%\fabricio{MANDAR PARA FORMALIZAÇÃO.}
%O emprego de \textit{setoids} neste projeto foi essencial para manter a formalização construtiva.
%A capacidade de representar o conjunto de termos $\alpha$-equivalentes, \textit{setoids} são essências para manter a formalização construtiva.
%Além do problema da reescrita, descrito acima, e  O axioma da extensionalidade funcional é necessário para demonstrar que o conjunto de funções é um conjunto de permutação (\fabricio{REFERENCIAL TEÓRICO}). Apesar do sistema formal do Coq ser consistente com o axioma, perde-se a construtividade, o que pode acarretar em atritos futuros, relacionados a extração de código verificado ou o uso da formalização como biblioteca. \textit{Setoids} permitem recuperar a extensionalidade funcional, mantendo a construtividade, através de uma nova relação de equivalência computacional para funções (\fabricio{VEJA FORMALIZAÇÃO.}).

\section{Exemplo de classes de tipo e reescrita generalizada}\label{sec:exemplos}
Com o intuito de exemplificar tudo que foi apresentado, abaixo apresento uma implementação da classe de tipos grupo:\\
\begin{minipage}[t]{0.5\linewidth}
\begin{coqcode*}{linenos=true}
Class Neutral A := neutral: A.
Class Operator A := op: A → A → A.
Class Inverse A := inv: A → A.

Class Group (A : Type)
  `{Ntr: Neutral A, Opr: Operator A, Inv: Inverse A, Equiv A} : Prop := {
	grp_setoid :> Equivalence (≡@{A});
	grp_op_proper :> Proper ((≡@{A}) ⇒ (≡@{A}) ⇒ (≡@{A})) (+);
	grp_inv_proper :> Proper ((≡@{A}) ⇒ (≡@{A})) (-);
	(* ... *)
	grp_left_inv : ∀ (x : A), (-x) + x ≡@{A} ɛ@{A};
}.
\end{coqcode*}
\end{minipage}
\hspace{-1em}
\begin{minipage}[t]{0.5\linewidth}
\begin{coqcode}
Notation ɛ := neutral.
Infix "+" := op.
Notation "- x" := (inv x).
\end{coqcode}
\end{minipage}\vspace*{.6em}\\
Linhas 1--3 são classes operacionais. O objetivo é dar um nome canônico, e notação, as operações e elementos de grupos. Temos o elemento neutro (\coqinline{neutral}  \coqinline{ɛ}), a operação binária (\coqinline{op}  \coqinline{+}) e função inversa (\coqinline{inv}  \coqinline{-}). Entre as linhas 5--12 tem-se a implementação da classe predicada \coqinline{Group} (a maior parte é omitida para simplificação da discussão), seu objetivo é agrupar os axiomas de grupo. A classe recebe cinco parâmetros, um explícito tipo \coqinline{A}, seguindo de quatro implícitos generalizados (entre \coqinline{`{}}): três instâncias das classes operacionais e por último uma instância da classe \coqinline{Equiv}. Esta última permite definir grupo para um \textit{setoid} de \coqinline{A}, separando em classe operacional, referenciada pelo nome \coqinline{equiv} com notação \coqinline{≡}, e classe predicada (\coqinline{Equiv}) contendo a relação de equivalência. Os parâmetros entre \coqinline{`{}}, como dito, são implicitamente generalizados, isto é, caso dependam de outros parâmetros implícitos, o Coq os generaliza e incluem na lista de parâmetros implícitos, simplificando o trabalho do usuário, pois este não precisa memorizar todos os parâmetros necessários a uma classe. Concluindo a parte sintática, as notações terminadas em \coqinline{@{A}} fornecem explicitamente parâmetros, que diferentemente seriam inferidos implicitamente. As linhas 7--11 são as propriedades da classe, onde apenas a 11 é referente aos axiomas de grupo (as demais foram omitidas). As propriedades entre 7--9 são necessárias a reescrita \textit{setoid}: \coqinline{grp_setoid} é uma prova de equivalência para a relação \coqinline{≡@{A}} (classe \coqinline{Equivalence}) e \coqinline{grp_op_proper} e \coqinline{grp_inv_proper} são provas de que a operação binária e inversão são próprias, isto é, contextos nos quais é seguro realizar reescrita generalizada. \coqinline{Proper} espera dois argumentos: uma assinatura e uma função. A assinatura descreve as relações de equivalência para as entradas e saída da função, isto é, \coqinline{Proper ((≡@{A}) ⇒ (≡@{B}) ⇒ (≡@{C})) g} é equivalente a:
\begin{equation*}\label{eq:proper}
	\forall (x y : A) (z w : B),~x \equiv_A y \rightarrow z \equiv_B w \rightarrow g(x,z) \equiv_C g(y,w)
\end{equation*}
Assim, o assistente sabe como reescrever \coqinline{x + y} para \coqinline{z + y}, dado uma prova de \coqinline{x ≡ z}. 

O mecanismo de reescrita generalizada simplifica bastante as provas, eliminando quase completamente as inconveniências envolvendo \textit{setoid}. Abaixo apresento um breve resultado utilizando as técnicas descritas acima, comparando uma propriedade de teoria de grupos e seu equivalente no assistente:
\vspace*{.4em}\\
%\hspace*{-1em}
\begin{minipage}[t]{0.5\linewidth}
	\begin{lema}
		Seja $x \in G$, tal que $G$ é um grupo. A função inversa é involutiva:
		\begin{align*}
			& & x& =\\
			\text{\textup{id. esquerda}}& & \varepsilon + x& =\\
			\text{\textup{inv. esquerda}}& & (-(-x) + -x) + x& =\\
			\text{\textup{associatividade}}& & -(-x) + (-x + x)& =\\
			\text{\textup{inv. esquerda}}& & -(-x) + \varepsilon& =\\
			\text{\textup{id. direita}}& & -(-x) &
		\end{align*}
	\end{lema}
\end{minipage}
\hspace{.7 em}
\begin{minipage}[t]{0.5\linewidth}
	\begin{coqcode}
Lemma grp_inv_involutive {G: Type}
  (x: G) `{Group G}: -(-x) ≡@{G} x.
Proof with auto.

		
rewrite <-(grp_left_id x) at 2;
rewrite <-grp_left_inv;
rewrite <-grp_assoc;
rewrite grp_left_inv;
rewrite grp_right_id...
		
Qed.
	\end{coqcode}
\end{minipage}\vspace*{.6em}\\
À esquerda tem-se a prova informal, enquanto à direita sua prova formal. O lema \coqinline{grp_inv_involutive} tem como parâmetros: um tipo \coqinline{G} (implícito), um termo \coqinline{x} de \coqinline{G} e uma prova implícita de que \coqinline{G} possui uma instância de grupo. Por estar implicitamente generalizada (entre \coqinline{`{}}), o Coq inclui, (implicitamente) todos os parâmetros de \coqinline{Group}. Portanto, temos acesso a um operador binário, uma função inversa, um elemento neutro e uma equivalência para \coqinline{G}, além das propriedades de grupo e reescrita nos operadores \coqinline{+} e \coqinline{-}. Sem o mecanismo de reescrita generalizada fornecido pelo Coq, a demonstração do lema seria, em sua maioria, aplicação de propriedades de infraestrutura, em outras palavras, 
permite uma formalização sucinta, próximo a prova informal, enquanto os detalhes ficam a cargo do assistente de provas.
%Como pode ser notado, não há necessidade de lidar diretamente com a implementação de \textit{setoids} e lemas de compatibilidade manualmente. Tornando a formalização quase transparente aos \textit{setoids}.
\chapter{Formalização}\label{chp:formalizacao}

% ----------------------------------------------------------
% Finaliza a parte no bookmark do PDF
% para que se inicie o bookmark na raiz
% e adiciona espaço de parte no Sumário
% ----------------------------------------------------------
\phantompart

% ---
% Conclusão
% ---
\chapter{Perspectivas Futuras}\label{chp:perspectivas}

% ---

% ----------------------------------------------------------
% ELEMENTOS PÓS-TEXTUAIS
% ----------------------------------------------------------
\postextual
% ----------------------------------------------------------

% ----------------------------------------------------------
% Referências bibliográficas
% ----------------------------------------------------------
\bibliography{ref}

% ----------------------------------------------------------
% Glossário
% ----------------------------------------------------------
%
% Consulte o manual da classe abntex2 para orientações sobre o glossário.
%
%\glossary

% ----------------------------------------------------------
% Apêndices
% ----------------------------------------------------------

% ---
% Inicia os apêndices
% ---
%\begin{apendicesenv}
%
%% Imprime uma página indicando o início dos apêndices
%\partapendices
%
%% ----------------------------------------------------------
%\chapter{Quisque libero justo}
%% ----------------------------------------------------------
%
%\lipsum[50]
%
%% ----------------------------------------------------------
%\chapter{Nullam elementum urna vel imperdiet sodales elit ipsum pharetra ligula
%ac pretium ante justo a nulla curabitur tristique arcu eu metus}
%% ----------------------------------------------------------
%\lipsum[55-57]
%
%\end{apendicesenv}
% ---


% ----------------------------------------------------------
% Anexos
% ----------------------------------------------------------

% ---
% Inicia os anexos
% ---
%\begin{anexosenv}
%
%% Imprime uma página indicando o início dos anexos
%\partanexos
%
%\chapter{Códigos}
%\begin{coqcode}
%Lemma aeq_perm_action1_cancel a b c m: b#(m) -> c#(m) -> (c ∙ b)ₜ (a ∙ c)ₜ m ≡α (a ∙ b)ₜ m.
%Proof.
%  destruct (a == b), (a == c), (b == c); subst; try (intros HB HC; autorewrite with nominal); auto;
%    try rewriten (action_switch b c); auto.
%  gen a b c; induction m as [| | t ? IH | ] using nominal_ind; try intros a b ? HB c ? ? HC; simpls.
%  - rewrite swap_cancel; fsetdec.
%  - constructor; [apply IHm1 | apply IHm2]; fsetdec.
%  - apply notin_remove_1 in HB as []; apply notin_remove_1 in HC as []; subst;
%      autorewrite with nominal; auto.
%    + rewriten (swap_neither a c b); auto; apply fresh AeqAbs;
%        rewriten (action_distr c b a c); rewriten (action_distr a b c b);
%          rewrite (action_switch c a); apply (IH [(_,_)]); auto; fsetdec.
%    + apply fresh AeqAbs; rewriten (action_distr c b a c);
%        rewriten (action_distr a b c b); apply (IH [(_,_)]); auto; fsetdec.
%    + destruct (a == t), (b == t), (c == t); subst; simpls; autorewrite with nominal; auto; try congruence.
%      * autorewrite with nominal; apply fresh AeqAbs with z.
%        rewriten (action_distr b z t b); rewriten (action_distr b z c b);
%          rewriten (action_distr b z t c); apply (IH [(_,_)]); auto; fsetdec.
%      * rewriten (swap_neither a c t); auto; apply fresh AeqAbs with z;
%          rewriten (action_distr c t a c); rewriten (action_distr a t c t);
%            rewrite (action_switch c a); apply (IH [(_,_)]); auto; fsetdec.
%      * autorewrite with nominal; apply fresh AeqAbs with z.
%        rewriten (action_distr t b a t); rewriten (action_distr a b t b);
%          apply (IH [(_,_)]); auto; fsetdec.
%      * autorewrite with nominal; apply fresh AeqAbs with z;
%          rewriten (action_distr t z c b); rewriten (action_distr t z a c);
%            rewriten (action_distr t z a b); apply (IH [(_,_)]); auto; fsetdec.
%  - intros d; intros; lets HB1:HB; lets HC1:HC;
%      apply notin_union_1 in HB1; apply notin_union_1 in HC1;
%        apply notin_remove_1 in HB1 as []; apply notin_remove_1 in HC1 as []; subst.
%    + congruence.
%    + autorewrite with nominal; rewriten (swap_neither d c b); auto;
%        apply AeqSub with ({{d}} ∪ {{b}} ∪ {{c}} ∪ support m1 ∪ support m2).
%      * intros z Hz; rewriten (action_distr d z d b); rewriten (action_distr z b d z).
%        rewriten (action_distr c z c b); rewriten (action_distr z b c z);
%          rewriten (action_distr z b d c); apply (H [(_,_)]); auto.
%      * apply (H0 nil); auto.
%    + autorewrite with nominal; apply AeqSub with ({{d}} ∪ {{b}} ∪ {{c}} ∪ support m1 ∪ support m2).
%      * intros z ?. rewriten (action_distr c z d b). rewriten (action_distr c b d c).
%        rewriten (action_distr d b c b). rewriten (action_distr d z c d).
%        rewriten (action_distr c z d z). rewriten (action_distr c z d b).
%        rewriten (action_distr d c d b). apply (H [(_,_)]); auto.
%      * apply (H0 nil); auto.
%    + apply AeqSub with ({{a}} ∪ {{d}} ∪ {{b}} ∪ {{c}} ∪ support m1 ∪ support m2).
%      * intros z ?. destruct (c == a), (b == a), (d == a); subst; autorewrite with nominal.
%        -- auto.
%        -- congruence.
%        -- auto.
%        -- rewriten (action_distr a z d b). rewriten (action_distr d z a b). rewriten (action_distr d z d a).
%           rewriten (action_distr z a d z). rewrite (action_switch z a).
%           apply (H [(_,_)]); auto.
%        -- congruence.
%        -- rewrite (swap_neither d c a); auto; autorewrite with nominal.
%           rewriten (action_distr c a d c). rewriten (action_distr d a c a).
%           rewrite (action_switch c d). apply (H [(_,_)]); auto; fsetdec.
%        -- rewriten (action_distr b z a b). rewriten (action_distr b z c b).
%           rewriten (action_distr b z a c). apply (H [(_,_)]); auto; fsetdec.
%        -- rewriten (action_distr a z d b). rewriten (action_distr a z c b).
%           rewriten (action_distr a z d c). apply (H [(_,_)]); auto.
%      * apply (H0 nil); auto.
%Qed.
%\end{coqcode}
%
%\end{anexosenv}

%---------------------------------------------------------------------
% INDICE REMISSIVO
%---------------------------------------------------------------------
\phantompart
\printindex
%---------------------------------------------------------------------

\end{document}
